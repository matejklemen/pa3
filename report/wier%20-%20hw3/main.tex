% ============================================================================================
% This is a LaTeX template used for the course
%
%  I M A G E   B A S E D   B I O M E T R I C S
%
% Faculty of Computer and Information Science
% University of Ljubljana
% Slovenia, EU
%
% You can use this template for whatever reason you like.
% If you have any questions feel free to contact
% ziga.emersic@fri.uni-lj.si
% ============================================================================================

\documentclass[9pt]{IEEEtran}

% basic
\usepackage[english]{babel}
\usepackage{graphicx,epstopdf,fancyhdr,amsmath,amsthm,amssymb,url,array,textcomp,svg,listings,hyperref,xcolor,colortbl,float,gensymb,longtable,supertabular,multicol,placeins,minted}

 % `sumniki' in names
\usepackage[utf8x]{inputenc}

 % search and copy for `sumniki'
\usepackage[T1]{fontenc}
\usepackage{lmodern}
\input{glyphtounicode}
\pdfgentounicode=1

% tidy figures
\graphicspath{{./figures/}}
\DeclareGraphicsExtensions{.pdf,.png,.jpg,.eps}

% correct bad hyphenation here
\hyphenation{op-tical net-works semi-conduc-tor trig-gs}

% ============================================================================================

\title{\vspace{0ex} %
% TITLE IN HERE:
Data processing, indexing and querying
\\ \normalsize{Web Information Extraction and Retrieval 2018/19, Faculty of Computer and Information Science, University of Ljubljana}}
\author{ %
% AUTHOR IN HERE:
Matej Klemen, Andraž Povše, Jaka Stavanja
\vspace{-4.0ex}
}

% ============================================================================================

\begin{document}

\maketitle

\begin{abstract}
In this work we present our implementation of a simple index and queries againts it.
First we are performing some data processing and indexing.
With the gathered data and structured index, we are going to perform data retrieval by testing multiple queries and displaying the results.
We test inverted index and naive approach and compare the performance of both.
\end{abstract}

\section{Introduction}
Querying the world wide web is a task we perform every day.
In the background, Google uses databases that contain indexed website data along with some other secrets to display most relevant results for the query at that specific moment.
Therefore, in order to display relevant results, we must first process the information on the website, build an index based on it, and then use this information to display most relevant results.
First step is processing the data and is described in chapter ~\ref{section:data_processing}.
Next comes the indexing, which is presented in chapter ~\ref{section:data_indexing}.
Final step in the puzzle is data retrieval we perform after we are faced with a query. The implementation is shown in chapter ~\ref{section:data_retrieval}

\section{Data processing}
\label{section:data_processing}
Separately describe data processing and indexing and data retrieval (with and without inverted index).
Tokenization, stopword removal, ...
\subsection{Inverted index}

\subsection{Naive approach}


\section{Data indexing}
\label{section:data_indexing}

Separately describe data processing and indexing and data retrieval (with and without inverted index).
Indexing process, index occurency of each word, ..
\subsection{Inverted index}

\subsection{Naive approach}

\section{Database}
Describe the database (number of indexed words, words and documents with the highest frequencies, …)

\section{Data retrieval}
\label{section:data_retrieval}

Separately describe data processing and indexing and data retrieval (with and without inverted index).

How we handled queries, searching words in our index, ...
\subsection{Inverted index}

\subsection{Naive approach}

\section{Conclusion}
\label{section:conclusion}
We presented 2 different approaches to handle a query.
First one used inverted index and was substantially faster than the other, naive implementation where we processed each file at time and had no built index.
Building the index (SQLite database) did not take that much time compared to 0 with naive approach, but the end result when measuring time elapsed for each query was much better. 
\end{document}